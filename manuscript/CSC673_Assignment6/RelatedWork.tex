\section{Related Works}
\subsection{The Attack on America and Civil Liberties Trade-Offs}
Since the formation of our dataset “The Attack on America and Civil Liberties Trade-Offs: A Three-Wave National Panel Survey, 2001-2004” has been collected ~\cite{data}, there have been over ten publications based on the data. Most publications are descriptive, where researchers are looking to explain patterns in the data, while few are methodological, where researchers are explaining steps taken during data collection. 

The most notable publications come directly from the original authors Davis and Silver. Davis and Silver study the willingness of individuals to give up civil liberties or security based on their ethnicity, political affiliation, trust in government, and sense of fear ~\cite{Davis_Silver_2004_1}. While they found many detailed relationships in the data, overall they discovered that Americans’ commitment to democratic values is highly contingent on a range of factors and that large-scale threats significantly impact an individual’s willingness to give up their rights. In a similar study ~\cite{Davis_2004_2}, Davis and Silver look specifically at the relationship between ethnicity and individuals’ responses to the 2001 September 11 attacks. Davis and Silver found that Latinos, Whites and African Americans all rallied around political leaders and the United States; however, each ethnicity had varying rates of supporting democratic values and evaluating the United States as partially responsible for the 2001 September 11 attacks. 

In addition, Davis and Silver analyze the dataset for political trends. They study what Americans perceived was the cause of the 2001 September attacks, where they found that 53 percent of Americans saw the United States as somewhat responsible~\cite{Davis_Silver_2004_3}. In ~\cite{Davis_Silver_2004_4}, Davis and Silver analyze whether the perceived threat surrounding terrorism from 2001 to 2004, influenced Americans' support of President George Bush. Ultimately, they conclude that the more individuals feared terrorism, the less likely they were to support Bush’s presidency.

While all these studies yield meaningful results about the data, there are no publications focused on using this dataset to identify untrustworthy survey responses. In our study, we focus on developing a classifier to identify untrustworthy survey responses to better provide higher fidelity data.

\subsection{Data Augmentation Methods}
Data augmentation refers to the techniques used to artificially increase the size of a dataset via a set of transformations applied to the original dataset. Data augmentation methods have recently been most popular within image and text data; however, they have been applied to a wide range of data types including tabular data. For image data, geometric, color and spatial transformation are often popular as it allows for one image to generate many samples ~\cite{Khoshgoftaar_2019}. 

Sampling techniques, generative models, and noise injection are popular for tabular data. Sampling techniques, such as SMOTE, aim to diversify the data by maintaining a balanced representation of the different classes ~\cite{SMOTE_Kegelmeyer_2002}. Generative models, such as Generative Adversarial Networks (GANs) or Variational Autoencoder (VAE), use neural networks to generate realistic artificial samples ~\cite{Meor_Yahaya_Teo_2023, Sandfort_Yan_2019}. In addition to generative modeling data augmentation, recently contrastive learning based and masked modeling based data augmentation methods have become popular for tabular data. VIME ~\cite{Yoon_2020} and SCRAF ~\cite{Bahri_2021} are examples of masked modeling based data augmentation methods and SubTab ~\cite{Ucar_2021} and Transtab ~\cite{Wang_Sun_2022} are contrastive learning based data augmentation methods. Some studies have started to use transformers for data augmentation such as FeatMix ~\cite{Chen_Yan_Chen_Wu_2023} and HiddenMix ~\cite{Chen_Yan_Chen_Wu_2023}. While these are the most relevant data augmentation techniques for our study, the field of data augmentation has a wide range of rapidly developing approaches. 