\begin{abstract}
% The title of the study is “Mitigating Batch Effects and Enhancing Cooperativeness Classification in Human Survey Data” and it is authored by Konghao Zhao and Cade Wiley from Wake Forest University. This study explores how to eliminate batch effects in a dataset and how to use data augmentation to improve the performance of classifiers, specifically to identify high fidelity data. The dataset used in this study is “The Attack on America and Civil Liberties Trade-Offs: A Three-Wave National Panel Survey, 2001-2004” produced by Darren Davis and Brain Silver. Ultimately we conclude that data augmentation is productive for improving the performance of classifiers to identify high fidelity data. The main contributions of this study are an automated framework for benchmarking data augmentation techniques, a framework using data augmentation techniques to generate high fidelity training data, and a classifier for predicting the cooperativeness of survey respondents.
The quality of a survey largely depends on how respondents respond to the question, and an inferior response generated by a lack of cooperation might severely deteriorates the quality of the survey. However, identifying subjectively whether a respondent is cooperative or not is a challenging problem. This study focuses on enhancing the classification of respondent cooperativeness in a survey data. To identify high fidelity data and improve the performance of classifiers, the study addresses the challenges of noisy, imbalanced, and biased data through well-structured data processing pipelines, augmentation methods, and a data transformation method. We conclude that data augmentation is productive for improving the performance of classifiers and our proposed data transformation could mitigate the batch effects. The contributions of this study are an automated framework for benchmarking data augmentation techniques and generating high fidelity training data, a good performing classifier for predicting the cooperativeness of survey respondents, and a data transformation method to mitigate the batch effect.
\end{abstract}
